\chapter{Projektmanagementmethode}\label{ch:projektmanagementmethode}
In diesem Kapitel ist die Projektmanagement-Methode «IPERKA» beschrieben, die während der Probe-IPA verwendet wird. Es werden die Gründe für diese Projektmanagement-Methode aufgeführt und eine alternative Methode mit Gründen, warum sie nicht verwendet wurde.

\section{IPERKA}\label{sec:METHODE}
Als Projektmanagement-Methode während der Probe-IPA wird IPERKA verendet. IPERKA ist eine Vorgehensmethode, die sich gut für Projekte mit überschaubarem Umfang und klar definierte Ziele eignet. Die Methode wirde bereits im ersten Lehrjahr in der Schule behandelt und ist durch das bereits bekannt. Der Name «IPERKA» setzt sich aus den Anfangsbuchstaben der sechs verschiedenen Schritten zusammen, nach denen vorgegangen wird:

\paragraph{I} nformieren: Den Auftrag verstehen, eine Vorstellung der Lösung erhalten, fehlende Informationen einholen, ordnen und bewerten
\paragraph{P} lanen: Nötige Arbeitsschritte definieren, einen Zeitplan erstellen, Methoden und Arbeitsmittel definieren
\paragraph{E} ntscheiden: Verschiedene Lösungsvarianten vergleichen, ausschlaggebende Kriterien definieren, eine Lösungsvariante auswählen
\paragraph{R} ealisieren: Arbeit umsetzen, Arbeitsschritte und Ergebnisse dokumentieren, auftrettende Probleme behandeln
\paragraph{K} ontrollieren: Arbeit testen, Resultate mit den Anforderungen vergleichen, Soll-Ist-Vergleich des Zeitplans, Dokumentation nochmals durchlesen
\paragraph{A} uswerten: Rückblick auf das Vorgehen, Arbeitsschritte beurteilen, Selbsteinschätzung vornehmen, mögliche Optimierungen für weitere Projekte definieren

\newpage
Die Aufteilung der Arbeit in die genannten Schritte unterstützt dabei, die Aufgaben sinnvoll zu strukturieren und systematisch vorzugehen. Ausserdem hilft die bewusste Steuerung des Arbeitsprozesses, die persönlichen Kompetenzen weiterzuentwickeln (vgl. ICT Berufsbildung Bern 2024\parencite{ict}). Die klare Trennung der Schritte stellt sicher, dass der Umfang und die erwarteten Ergebnisse der Aufgabe sich im Verlauf der Arbeit nicht mehr stark ändern, da IPERKA grundsätzlich kein «Rückwärtsgehen» in den Phasen, wie dies aus iterativen Modellen bekannt ist, vorsieht. Ausserdem stellt sie sicher, dass die Realisierung nicht zu schnell in Angriff genommen wird.

\section{Alternative Methode}\label{sec:alternative-methode}
Neben IPERKA wurde noch eine andere Vorgehensmethode angeschaut. Diese ist folgend, jeweils mit einer Begründung, wieso IPERKA der Methode vorgezogen wurde, kurz beschrieben.

\paragraph{Scrum} ist eine bekannte agile Vorgehensmehtode, die heutzutage in der Softwareentwicklung weit verbreitet ist. Die Methode legt den Fokus mehr auf Punkte wie laufende Software, gute Zusammenarbeit, oder flexibles Reagieren auf Veränderungen, ansttt einem strikten Plan zu folgen. Scrum sieht einen iterativen Prozess vor, der laufend optimiert werden soll, und definiert verschiedene Rollen, die jeweils ihre Aufgabe in diesem Scrum-Prozess haben. Ein «agiles» Vorgehen ist in der Softwareentwicklung grundsätzlich sinnvoll, da oft nicht von Beginn her klar ist, wie das Resultat schlussendlich aussehen soll. Da die Probe-IPA schlussendlich aber eine Prüfung ist, sind die Vorgaben und erwarteten Resultate relativ klar. Auch der Umfang der Probe-IPA und das Enddatum sind von Beginn an bekannt und verändern sich nicht während der Arbeit. Ausserdem wird die Aufgabe von nur einer Person bearbeitet, wofür die Abläufe von Scrum nicht optimal geeignet sind. Aus diesen Gründen wird IPERKA gegenüber Scrum vom Lernenden bevorzugt.
