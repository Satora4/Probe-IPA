\chapter{Arbeitsumgebung}\label{ch:arbeitsumgebung}
In diesem Kapitel wird beschrieben, wie die Arbeitsumgebung des Lernenden während der Probe-IPA aussah.


\section{Arbeitsplatz}\label{sec:arbeitsplatz}
\subsection{Office Arbeitsplatz}\label{subsec:office-arbeitsplatz}
Die Probe-IPA wird am gewohnten Arbeitsplatz im Fünferbüro des Lernenden durchgeführt. Als Arbeitsgerät wird ein Notebook verwendet, welches mithilfe einer Dockingstation das Gerät mit zwei Monitoren und dem Firmennetzwerk verbindet. Der Stuhl und Tisch sind höhenverstellbar, und der Lernende kann dadurch in verschiedenen Sitzpositionen oder stehend arbeiten.

\begin{figure}[H]
    \begin{center}
        \includegraphics[width=0.8\textwidth]{ressourcen/Arbeitsplatz-Joel-Vontobel}
        \caption[Arbeitsplatz des Lernenden]{Arbeitsplatz des Lernenden}\label{fig:Arbeitsplatz-Joel-Vontobel}
    \end{center}
\end{figure}

\newpage
\section{Verwendete Tools}\label{sec:verwendete-tools}
Die folgende Tabelle zeigt auf, welche Tools für die Umsetzung der Probe-IPA eingesetzt wurden.

\renewcommand{\arraystretch}{1.5}
\begin{longtable}{|p{.22\textwidth}|p{.40\textwidth}|p{.38\textwidth}|}
    \hline
    \textbf{Tool}                    & \textbf{Einsatzzweck}                              & \textbf{Link}                                                             \\ \hline
    IntelliJ                         & Entwicklungsumgebung für die Programmierung        & \url{https://www.jetbrains.com/de-de/idea/}                               \\ \hline
    Docker                           & Ausführen der Programme                            & \url{https://www.docker.com/}                                             \\ \hline
    Git                              & Versionierung vom Quellcode                        & \url{https://git-scm.com/}                                                \\ \hline
    Postman                          & Ausführen von HTTP-Requests (Testen vom Backend)   & \url{https://www.postman.com/}                                            \\ \hline
    Bitbucket                        & Speicherung der Quellcodes                         & \url{https://bitbucket.org/product/}                                      \\ \hline
    Jenkins                          & Tool für Pipelines, um Tests und die Codequalität automatisch zu prüfen  & \url{https://www.jenkins.io/}                       \\ \hline
    Confluence                       & Abbildung der Probe-IPA-Kriterien                  & \url{https://www.atlassian.com/de/software/confluence}                    \\ \hline
    Github                           & Sicherung und Versionirung der Dokumentation       & \url{https://github.com/}                                                 \\ \hline
    Draw.io                          & Erstellen von Diagrammen und Abbildungen           & \url{https://www.drawio.com/}                                             \\ \hline
    TexStudio                        & Dokumentationstool                                 & \url{https://www.texstudio.org/}                                          \\ \hline
    LaTeX                            & Ein Dokumentenvorbereitungssystem                  & \url{https://www.latex-project.org/}                                      \\ \hline
    Mattermost                       & Text basiertes Kommunikationsmittel                & \url{https://mattermost.com/}                                             \\ \hline
    Microsoft Teams                  & Video basiertes Kommunikationsmittel               & \url{https://www.microsoft.com/de-ch/microsoft-teams/group-chat-software} \\ \hline
    Microsoft Excel                  & Erstellung und Bearbeitung des Zeitplans           & \url{https://www.microsoft.com/de-ch/microsoft-365/excel?market=ch}       \\ \hline
    LanguageTool                  	 & Prüfung der Rechtschreibung des Dokuments          & \url{https://languagetool.org/de}       								  \\ \hline
\end{longtable}
\renewcommand{\arraystretch}{1}
\newpage