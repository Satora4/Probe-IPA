\chapter{Versionierung und Sicherung der Arbeitsergebnisse}\label{ch:versionierung-und-sicherung-der-arbeitsergebnisse}
In diesem Kapitel wird beschrieben, wie der Lernende sicherstellt, dass die erarbeiteten Ergebnisse während der Probe-IPA sicher gespeichert und jederzeit wieder aufrufbar sind. Die Versionierung soll es ermöglichen, frühere erstellte Versionen der Daten jederzeit wiederherstellen zu können. Die Massnahmen werden hier vom Lernenden aufgeführt.

\section{Verwendung von Git zu Versionierung}
Für die Versionierung von der Probe-IPA wird Git verwendet. Git ist ein Versionierungstool und wird genutzt, um Quellcode zu versionieren und zu beschriften. In der Schule so wie auch in der Firma wurde Git bereits in diversen Projekten verwendet, um den Quellcode übersichtlich zu versionieren und in der Cloud zu sichern. Mit Git kann man sogenannte «Commits» machen, um einen kleinen Teil der Änderungen zu speichern und zu beschriften. Diese Änderungen kann man jederzeit wieder rückgängig machen oder aufrufen, um eine bestimmte Version genauer zu analysieren. Durch diese Commits ist der Quellcode für eine andere Person verständlicher zu lesen.

\section{Quellcode}
Der Quellcode der eingehenden Message-Queue-Nachrichten im Web-GUI wird mit Git verwaltet und ist in einem Repository auf Bitbucket gespeichert. In Abbildung 5.1 ist die Git Commit History des Quellcodes ersichtlich.

\begin{figure}[H]
	\begin{center}
		\includegraphics[width=0.8\textwidth]{ressourcen/placeholder}
		\caption[Git Commit History des Quellcodes]{Git Commit History des Quellcodes}\label{fig:Git-Commit-History-des-Quellcodes}
	\end{center}
\end{figure}

\section{Probe-IPA Dokumentation}
Die Probe-IPA-Dokumentation wird mithilfe von \LaTeX geschrieben, wodurch eine Versionierung mit Git auch möglich wird. Die \LaTeX- und alle anderen benötigten Dateien werden auf ein privates Repository in der Cloud geladen. In Abbildung 5.2 ist die Git Commit History der Probe-IPA-Dokumentation ersichtlich.

\begin{figure}[H]
	\begin{center}
		\includegraphics[width=0.8\textwidth]{ressourcen/placeholder}
		\caption[Git Commit History der Dokumentation]{Git Commit History der Dokumentation}\label{fig:Git-Commit-History-der-Dokumentation}
	\end{center}
\end{figure}