\chapter{Arbeitsprotokoll}\label{ch:arbeitsprotokoll}
\renewcommand{\arraystretch}{1.5}

\begin{longtable}{p{.22\textwidth}|p{.78\textwidth}}
	\hline
	\textbf{Datum}                       & 06.11.2024            \\
	\hline
	\textbf{Bearbeitete Arbeitspakete}   & 7.1, 7.2, 7.3, 7.4                  \\
	\hline
	\textbf{Arbeitszeit}                 & 8h                                    \\
	\hline
	\textbf{Überzeit}                    & 0h                                    \\
	\hline
	\textbf{Vergleich mit dem Zeitplan}  & Zeitplan noch nicht fertig \\
	\hline
	\textbf{Erfolge und Probleme} & Durch die Vorlage des Dokuments und \LaTeX konnte ich direkt mit dem ersten Teil der Dokumentation beginnen, was mir viel Zeit erspart hat. Ich habe heute mein Ziel, den ersten Teil grösstenteils abzuschliessen, erreicht und hatte keine grösseren Probleme, die aufgetreten sind.
	\\
	\hline
	\textbf{Tagesreflexion} & Ich bin heute gut in die Probe-IPA gestartet. Anfangs war ich ein wenig überfordert und wusste nicht, wo ich anfangen sollte. Nach der ersten Stunde hat sich das aber wieder gelegt und ich konnte konzentriert an meinem Ziel arbeiten.
	\\
	\hline
	\textbf{In Anspruch genommene Hilfe} & Keine                              \\
	\hline
\end{longtable}\label{tab:arbeitsprotokoll-06.11.2024}
\newpage

\begin{longtable}{p{.22\textwidth}|p{.78\textwidth}}
	\hline
	\textbf{Datum}                       & 07.11.2024            \\
	\hline
	\textbf{Bearbeitete Arbeitspakete}   & 1.1, 1.2, 2.1, 2.2                  \\
	\hline
	\textbf{Arbeitszeit}                 & 8h                                    \\
	\hline
	\textbf{Überzeit}                    & 0h                                    \\
	\hline
	\textbf{Vergleich mit dem Zeitplan}  & Ich konnte alle geplanten Arbeiten von heute erledigen und hatte auch eine Stunde übrig. Ich habe also bereits mit der Aufgabe 2.3 angefangen. \\
	\hline
	\textbf{Erfolge und Probleme} & Heute konnte ich mit dem 2. Teil der Dokumentation anfangen. Die Phase Informieren konnte ich zeitgerecht abschliessen und habe bereits in der Phase Planen die Arbeitspakete und den Zeitplan fertiggestellt. Momentan habe ich mit der Aufgabe \ref{tab:planen-2.3} begonnen, die für morgen eingeplant ist.
	\\
	\hline
	\textbf{Tagesreflexion} & Ich konnte heute motiviert in den Tag starten, da ich gestern meine Tagesziele erreicht habe. Ich konnte mich nicht wirklich für \ref{tab:informieren-1.1} und \ref{tab:informieren-1.2} motivieren, wusste aber, dass ich mich danach mit den Arbeitspaketen und dem Zeitplan beschäftigen kann. Diese zwei Teile haben mir Spass gemacht, da ich mich danach am Zeitplan orientieren kann.
	\\
	\hline
	\textbf{In Anspruch genommene Hilfe} & Ich habe mich bei Loris Diana \ref{ch:projektaufbauorganisation} erkundigt, ob ich die Codequalität mit dem Tool, welches mein Projekt nutzt, überprüfen darf oder ich selbst diese Überprüfung machen muss.                                \\
	\hline
\end{longtable}\label{tab:arbeitsprotokoll-07.11.2024}
\newpage

\begin{longtable}{p{.22\textwidth}|p{.78\textwidth}}
	\hline
	\textbf{Datum}                       & 08.11.2024            \\
	\hline
	\textbf{Bearbeitete Arbeitspakete}   & 2.3, 2.4                  \\
	\hline
	\textbf{Arbeitszeit}                 & 8h                                    \\
	\hline
	\textbf{Überzeit}                    & 0h                                    \\
	\hline
	\textbf{Vergleich mit dem Zeitplan}  & Ich konnte alle geplanten Arbeitspakete machen oder anfangen.  \\
	\hline
	\textbf{Erfolge und Probleme} & Heute hatte ich Probleme mit der Konzentration. Sie ist schwungartig gekommen und wieder gegangen. Durch das habe ich ein wenig Zeit verloren und bin jetzt wieder im Zeitplan. Mein Ziel für heute war eigentlich, das Arbeitspaket 2.5 fertig zu haben, sodass ich nächste Woche mit dem Testkonzept starten kann. Ich habe es nicht ganz geschafft. Es fehlt aber nicht mehr viel.
	\\
	\hline
	\textbf{Tagesreflexion} & Wie bei «Erfolge und Probleme» schon erwähnt, fehlte mir heute ein wenig die Konzentration. Ich glaube, es könnte mit der VA zusammenhängen, da ich diese Woche noch daran gearbeitet habe und ich so mehrheitlich am Schreiben war. Ansonsten kann ich mich nicht beklagen, da ich immer noch ein kleines bisschen im Vorsprung zum Zeitplan bin.
	\\
	\hline
	\textbf{In Anspruch genommene Hilfe} & Ich habe mich heute bei Dominic Monzón \ref{ch:projektaufbauorganisation} erkundigt, ob für die Filterung alle MQ\_IN\_STATI gebraucht werden, weil im Quellcode die einen mit «// TODO huberdav: can be removed eventually» beschriftet sind.                                \\
	\hline
\end{longtable}\label{tab:arbeitsprotokoll-08.11.2024}
\newpage

\begin{longtable}{p{.22\textwidth}|p{.78\textwidth}}
	\hline
	\textbf{Datum}                       & 12.11.2024            \\
	\hline
	\textbf{Bearbeitete Arbeitspakete}   & 2.5                  \\
	\hline
	\textbf{Arbeitszeit}                 & 3.5h                                    \\
	\hline
	\textbf{Überzeit}                    & -0.5h                                    \\
	\hline
	\textbf{Vergleich mit dem Zeitplan}  & Momentan bin ich auf Kurs mit dem Zeitplan und habe keine Abweichungen. \\
	\hline
	\textbf{Erfolge und Probleme} & Alle Lösungskonzepte sind fertig und ich konnte mit den Testkonzepten anfangen.
	\\
	\hline
	\textbf{Tagesreflexion} & Heute kam ich gut voran und bin immer noch im Zeitplan. Der Vorsprung von letzter Woche ist heute jedoch nicht mehr vorhanden, aber ich bin nicht im Verzug. Heute war die Konzentration wieder hoch und ich hoffe, das bleibt auch für die bleibenden Tage noch weiter so.
	\\
	\hline
	\textbf{In Anspruch genommene Hilfe} & Ich habe mich bei Loris Diana \ref{ch:projektaufbauorganisation} erkundigt, ob ein Integration-Test für einen Endpoint eine Datenbank braucht, die läuft. Er hat mir erklärt, dass es eine braucht, aber auch eine In-Memory-Datenbank ausreicht.                              \\
	\hline
\end{longtable}\label{tab:arbeitsprotokoll-12.11.2024}
\newpage

\begin{longtable}{p{.22\textwidth}|p{.78\textwidth}}
	\hline
	\textbf{Datum}                       & 13.11.2024            \\
	\hline
	\textbf{Bearbeitete Arbeitspakete}   & 2.6, 3.1, 4.1                  \\
	\hline
	\textbf{Arbeitszeit}                 & 8.5h                                    \\
	\hline
	\textbf{Überzeit}                    & 0.5h                                    \\
	\hline
	\textbf{Vergleich mit dem Zeitplan}  & Durch die nicht verwendeten 3 Stunden von der Entscheidung wurde ich heute bei dem Punkt 4.1 zwei Stunden früher als geplant fertig. \\
	\hline
	\textbf{Erfolge und Probleme} & Heute konnte ich nach vielen Tagen Dokumentation schreiben, endlich mit dem Programmieren anfangen. Ich habe den Endpunkt bereits fertig und Dokumentiere diesen jetzt.
	\\
	\hline
	\textbf{Tagesreflexion} & Mit viel Motivation startete ich heute in den Tag, da ich wusste, dass ich endlich mit dem Programmieren anfangen kann. Ich habe das Textkonzept fertig geschrieben und konnte ziemlich schnell die Entscheidung abschliessen und habe so zwei Meilensteine in meiner Probe-IPA erreicht.
	\\
	\hline
	\textbf{In Anspruch genommene Hilfe} & Ich hatte Probleme mit einem Bean in Java, welches für den MqInService gebraucht wurde. Dieses wurde nicht gefunden und Dominic Monzón \ref{ch:projektaufbauorganisation} hat mir geholfen, dieses für den Service zu erstellen. Ausserdem konnte ich den Endpoint nicht ansprechen, da die Url fälschlicherweise falsch war und habe mir Hilfe von Micha Schena geholt (Ein Mitarbeiter im CardX-Team).                              \\
	\hline
\end{longtable}\label{tab:arbeitsprotokoll-13.11.2024}
\newpage

\begin{longtable}{p{.22\textwidth}|p{.78\textwidth}}
	\hline
	\textbf{Datum}                       & 14.11.2024            \\
	\hline
	\textbf{Bearbeitete Arbeitspakete}   & 4.2                  \\
	\hline
	\textbf{Arbeitszeit}                 & 8h                                    \\
	\hline
	\textbf{Überzeit}                    & 0h                                    \\
	\hline
	\textbf{Vergleich mit dem Zeitplan}  & Momentan bin ich synchron mit dem Zeitplan. \\
	\hline
	\textbf{Erfolge und Probleme} & Heute konnte ich die Mindestanforderungen abschliessen und mit der Erweiterung weiter machen.
	\\
	\hline
	\textbf{Tagesreflexion} & Ich konnte heute wieder mit viel Motivation in den Tag starten, da heute die Implementierung im Frontend bevor stand. Ich hatte länger als gedacht, da ich gestern einen Endpoint vergessen habe zu implementieren, welcher ich heute nacharbeiten musste.
	\\
	\hline
	\textbf{In Anspruch genommene Hilfe} & Loris Diana \ref{ch:projektaufbauorganisation} hat mir heute erklärt, wie die Generierung von den HTTP-Requests im Frontend genau funktioniert.                              \\
	\hline
\end{longtable}\label{tab:arbeitsprotokoll-14.11.2024}
\newpage

\begin{longtable}{p{.22\textwidth}|p{.78\textwidth}}
	\hline
	\textbf{Datum}                       & 15.11.2024            \\
	\hline
	\textbf{Bearbeitete Arbeitspakete}   & 4.3, 4.5                  \\
	\hline
	\textbf{Arbeitszeit}                 & 8h                                    \\
	\hline
	\textbf{Überzeit}                    & 0h                                    \\
	\hline
	\textbf{Vergleich mit dem Zeitplan}  & Da ich die Implementation von 4.4 \ref{tab:realisieren-4.4} pausiert habe, bin ich wieder synchron mit dem Zeitplan. \\
	\hline
	\textbf{Erfolge und Probleme} & Heute hatte ich viele Schwierigkeiten mit der Implementierung der Erweiterung im Frontend. Trotz des Beispiels von «Fehlgeschlagene Zahlung», schaffte ich es nicht, ein Filterergebnis im Frontend anzeigen zu lassen. Ich habe vieles ausprobiert und werde, falls noch Zeit ist, mit einem Fachverantwortlichen das Problem nochmals angehen.
	\\
	\hline
	\textbf{Tagesreflexion} & Ich hatte heute viel Motivation, um die Implementierung für den grössten Teil zu ermöglichen. Leider schwand diese Motivation ziemlich schnell während der Implementierung der Erweiterung im Frontend. Anfangs kam ich noch gut voran. Mit der Zeit wusste ich aber nicht, ob diese Implementierung funktionieren wird oder nicht. Da ich für beide Implementationen (das Backend und Frontend der Erweiterung) noch keine Dokumentation geschrieben habe, habe ich mich dazu entschieden, die Implementation zu pausieren und damit anzufangen, sodass ich nicht noch mehr in den Verzug komme.
	\\
	\hline
	\textbf{In Anspruch genommene Hilfe} & Dominic Moncón \ref{ch:projektaufbauorganisation} hat mir heute eine Frage bezüglich der Filterung beantwortet. Für die Filterung von Nachrichten konnte ich nicht auf die entsprechende Spalte zugreifen, und die Frage war, wie ich jetzt auf sie zugreifen kann. Bei anderen Feldern gibt es eine Variable, welche dies ermöglicht. Oberhalb der Felder steht aber, dass sie generiert wurden, und ich wusste nicht, ob ich sie direkt in dieser Klasse hinzufügen kann oder ob ich anderswo eine kleine Änderung unternehmen muss. Ich konnte schlussendlich einfach diese Variable hinzufügen.                           \\
	\hline
\end{longtable}\label{tab:arbeitsprotokoll-15.11.2024}
\newpage

\begin{longtable}{p{.22\textwidth}|p{.78\textwidth}}
	\hline
	\textbf{Datum}                       & 19.11.2024            \\
	\hline
	\textbf{Bearbeitete Arbeitspakete}   & 5.1 (angefangen)                  \\
	\hline
	\textbf{Arbeitszeit}                 & 3h                                    \\
	\hline
	\textbf{Überzeit}                    & -1h                                    \\
	\hline
	\textbf{Vergleich mit dem Zeitplan}  & Im Moment bin ich noch im Zeitplan. \\
	\hline
	\textbf{Erfolge und Probleme} & Heute konnte ich die Phase Realisieren abschliessen und mit der Implementierung der Tests anfangen. 
	\\
	\hline
	\textbf{Tagesreflexion} & Ich habe heute noch die Mock Implementation vom MqInService gemacht für das Akzeptanzkriterium. Dies ging ziemlich schnell und war nicht schwer zu implementieren. Ausserdem habe ich den ersten Test heute geschrieben, ohne grosse Probleme. Momentan kommen aber die hinzugefügten Daten in der Datenbank noch nicht im Test an. Dieses Problem werde ich aber morgen in Angriff nehmen. 
	\\
	\hline
	\textbf{In Anspruch genommene Hilfe} & -                              \\
	\hline
\end{longtable}\label{tab:arbeitsprotokoll-19.11.2024}
\newpage

\begin{longtable}{p{.22\textwidth}|p{.78\textwidth}}
	\hline
	\textbf{Datum}                       & 20.11.2024            \\
	\hline
	\textbf{Bearbeitete Arbeitspakete}   & 5.1, 5.2 (angefangen), 7.4                  \\
	\hline
	\textbf{Arbeitszeit}                 & 9h                                    \\
	\hline
	\textbf{Überzeit}                    & 1h                                    \\
	\hline
	\textbf{Vergleich mit dem Zeitplan}  & Durch die Verzögerung mit der Prüfung der Codequalität bin ich momentan 4 Stunden im Verzug.\\
	\hline
	\textbf{Erfolge und Probleme} & Heute konnte ich das Schreiben der Tests abschliessen. Leider hatte ich Probleme bei der Codequalität und bin so heute nicht mehr zum Finalisieren der Dokumentation gekommen.
	\\
	\hline
	\textbf{Tagesreflexion} & Heute war ein langer Tag durch das Aufholen der Stunde von gestern. Ich war motiviert, die Tests endlich zu implementieren und freute mich, mit der Codequalität anzufangen. Leider war ich damit nicht so schnell fertig wie gedacht und bin jetzt im Verzug. Ich rechne aber damit, dass ich mit der Finalisierung der Dokumentation nicht so lange brauche wie im Zeitplan beschrieben, um so den Verzug wiedergutzumachen.
	\\
	\hline
	\textbf{In Anspruch genommene Hilfe} & Dominic Moncón \ref{ch:projektaufbauorganisation} hat mir heute bei Problemen mit dem Build (Codequalität prüfen) geholfen, da es einige Schwierigkeiten gab, die ich nicht selbst lösen konnte.                            \\
	\hline
\end{longtable}\label{tab:arbeitsprotokoll-20.11.2024}
\newpage

\begin{longtable}{p{.22\textwidth}|p{.78\textwidth}}
	\hline
	\textbf{Datum}                       & 21.11.2024            \\
	\hline
	\textbf{Bearbeitete Arbeitspakete}   & 5.3, 6.1                  \\
	\hline
	\textbf{Arbeitszeit}                 & 8h                                    \\
	\hline
	\textbf{Überzeit}                    & 0h                                    \\
	\hline
	\textbf{Vergleich mit dem Zeitplan}  & Momentan habe ich einen kleinen Vorsprung gegenüber dem Zeitplan da die Finalisierung des Dokuments nicht so lange gedauert hat wie geplant. Durch das konnte ich heute schon die Kurzfassung schreiben. \\
	\hline
	\textbf{Erfolge und Probleme} & Heute konnte ich das Dokument finalisieren und habe somit den Meilenstein Kontrollieren erreicht.
	\\
	\hline
	\textbf{Tagesreflexion} & Heute was ich sehr konzentriert bei der Überarbeitung der Doku, um dies heute noch fertig zu bekommen. Ich wolle nicht noch morgen im Verzug sein und habe dies auch heute geschafft.
	\\
	\hline
	\textbf{In Anspruch genommene Hilfe} & Ich war bei der Codequalität nicht ganz sicher, welches Bild ich nehmen soll und habe darum Loris Diana \ref{ch:projektaufbauorganisation} um Hilfe bei der Auswahl gebeten, da ich nicht sicher war, welches Bild mehr Bedeutung in der Dokumentation zeigen wird.                           \\
	\hline
\end{longtable}\label{tab:arbeitsprotokoll-21.11.2024}
\newpage

\begin{longtable}{p{.22\textwidth}|p{.78\textwidth}}
	\hline
	\textbf{Datum}                       & 22.11.2024            \\
	\hline
	\textbf{Bearbeitete Arbeitspakete}   & 6.2, 7.5                  \\
	\hline
	\textbf{Arbeitszeit}                 & 8h                                    \\
	\hline
	\textbf{Überzeit}                    & 0h                                    \\
	\hline
	\textbf{Vergleich mit dem Zeitplan}  & Ich wurde mit dem Anhang ein wenig früher Ferig und habe jetzt noch ein bisschen debugging bezüglich Erweiterung Frontend gemacht.  \\
	\hline
	\textbf{Erfolge und Probleme} & Heute war der letzte Tag der Probe-IPA und ich habe fast alles erfolgreich erledigen können. Am Schluss hatte ich noch ein wenig Zeit und habe mich nochmals an das Debugging gesetzt und ich konnte es tatsächlich lösen.
	\\
	\hline
	\textbf{Tagesreflexion} & Ich freue mich das die Probe-IPA wieder zu Ende geht, aber ich hatte auch heute noch Spass mit den letzten zwei Arbeitspaketen. Insgesamt bin ich sehr zufrieden, was ich in diesem Zehn Tagen erreicht habe.
	\\
	\hline
	\textbf{In Anspruch genommene Hilfe} & -                              \\
	\hline
\end{longtable}\label{tab:arbeitsprotokoll-22.11.2024}
\newpage
