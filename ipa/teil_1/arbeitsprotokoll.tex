\chapter{Arbeitsprotokoll}\label{ch:arbeitsprotokoll}
\renewcommand{\arraystretch}{1.5}

\begin{longtable}{p{.22\textwidth}|p{.78\textwidth}}
    \hline
    \textbf{Datum}                       & ...            \\
    \hline
    \textbf{Bearbeitete Arbeitspakete}   & ...                  \\
    \hline
    \textbf{Arbeitszeit}                 & ...                                    \\
    \hline
    \textbf{Überzeit}                    & ...                                    \\
    \hline
    \textbf{Vergleich mit dem Zeitplan}  & ... \\
    \hline
    \textbf{Erfolge und Probleme} & ...
    \\
    \hline
    \textbf{Tageasdfsreflexion} & ...
    \\
    \hline
    \textbf{In Anspruch genommene Hilfe} & ...                              \\
    \hline
\end{longtable}\label{tab:arbeitsprotokoll-...}

\newpage
\begin{longtable}{p{.22\textwidth}|p{.78\textwidth}}
	\hline
	\textbf{Datum}                       & 06.11.2024            \\
	\hline
	\textbf{Bearbeitete Arbeitspakete}   & 7.1, 7.2, 7.3                  \\
	\hline
	\textbf{Arbeitszeit}                 & 8h                                    \\
	\hline
	\textbf{Überzeit}                    & 0h                                    \\
	\hline
	\textbf{Vergleich mit dem Zeitplan}  & Zeitplan noch nicht fertig \\
	\hline
	\textbf{Erfolge und Probleme} & Durch die Vorlage des Dokuments und \LaTeX konnte ich direkt mit dem ersten Teil der Dokumentation beginnen, was mir viel Zeit erspart hat. Ich habe heute mein Ziel, den ersten Teil grösstenteils abzuschliessen, erreicht und hatte keine grösseren Probleme die aufgetretten sind.
	\\
	\hline
	\textbf{Tagesreflexion} & Ich bin heute gut in die Probe-IPA gestartet. Anfangs war ich ein wenig übervordert und wusste nicht wo ich anfangen sollte. Nach der ersten Stunde hat sich das aber wieder gelegt und ich konnte konzentriet an meinem Ziel arbeiten.
	\\
	\hline
	\textbf{In Anspruch genommene Hilfe} & Keine                              \\
	\hline
\end{longtable}\label{tab:arbeitsprotokoll-06.11.2024}
\newpage
