\section{Testfälle}
Alle automatischen Tests werden auf 25 Einträge limitiert. Sie haben alle den Fehlerzustand MQ\_IN\_STATUS und sind absteigend sortiert nach MODIFIED\_AT. Die erhaltenen Einträge beinhalten nur die Spalten, die vorgegeben sind.

\subsection{Testfälle der Mindestanforderungen}

\begin{longtable}{p{.3\textwidth}|p{.65\textwidth}}
	\hline
	\textbf{Testfall}               & \textbf{M1} \\
	\hline
	\textbf{Beschreibung}   		& Datenbank-Abfrage für die Erstellung von der Tabelle \\
	\hline
	\textbf{Art}    				& Integration-Test \\
	\hline
	\textbf{Vorbedingungen}    		& Das Backend läuft und eine passende Datenbank existiert. \\
	\hline
	\textbf{Konfiguration}   	 	& 
	\begin{itemize}
		\item Die Daten für den Aufruf werden erstellt.
		\item Die erwarteten Daten werden erstellt.
	\end{itemize} \\
	\hline
	\textbf{Ablauf}    				& 
	\begin{enumerate}
		\item Die Datenbank-Abfrage wird durchgeführt.
		\item Die erhaltenen Daten werden mit den bereits erstellten Daten überprüft.
	\end{enumerate} \\
	\hline
	\textbf{Erwartetes Ergebnis}    & Es gibt keine Unterschiede zwischen den Daten aus der Datenbank und den erwarteten Daten.  \\
	\hline
\end{longtable}\label{tab:testfall-M1}

\begin{longtable}{p{.3\textwidth}|p{.65\textwidth}}
	\hline
	\textbf{Testfall}               & \textbf{M2} \\
	\hline
	\textbf{Beschreibung}   		& Neuer Verarbeitungsversuch starten \\
	\hline
	\textbf{Art}    				& Integration-Test \\
	\hline
	\textbf{Vorbedingungen}    		& Das Backend läuft und eine passende Datenbank existiert. \\
	\hline
	\textbf{Konfiguration}   	 	& 
	\begin{itemize}
		\item Die Daten für den Aufruf werden erstellt.
	\end{itemize} \\
	\hline
	\textbf{Ablauf}    				& 
	\begin{enumerate}
		\item Die Datenbank-Abfrage wird durchgeführt.
		\item Die erhaltenen Daten werden überprüft.
	\end{enumerate} \\
	\hline
	\textbf{Erwartetes Ergebnis}    & Ein Objekt auf der Datenbank wurde überschrieben. Das überschriebene Objekt hat einen Status von NEW.  \\
	\hline
\end{longtable}\label{tab:testfall-M2}

\begin{longtable}{p{.3\textwidth}|p{.65\textwidth}}
	\hline
	\textbf{Testfall}               & \textbf{M3} \\
	\hline
	\textbf{Beschreibung}   		& Mapping von DAO zu DTO \\
	\hline
	\textbf{Art}    				& Unit-Test \\
	\hline
	\textbf{Vorbedingungen}    		& Das Backend läuft. \\
	\hline
	\textbf{Konfiguration}   	 	& 
	\begin{itemize}
		\item Ein DAO-Objekt ist erstellt und gefüllt mit Daten.
	\end{itemize} \\
	\hline
	\textbf{Ablauf}    				& 
	\begin{enumerate}
		\item Die Funktion mapDaoToDto() ausführen, mit dem DAO-Objekt als Parameter.
		\item Der erhaltene Wert wird mit dem DAO-Objekt verglichen.
	\end{enumerate} \\
	\hline
	\textbf{Erwartetes Ergebnis}    & Es gibt keine Unterschiede zwischen dem generierten DTO- und dem bereits existierenden DAO-Objekt. \\
	\hline
\end{longtable}\label{tab:testfall-M3}

\begin{longtable}{p{.3\textwidth}|p{.65\textwidth}}
	\hline
	\textbf{Testfall}               & \textbf{M4} \\
	\hline
	\textbf{Beschreibung}   		& Mapping von einer Liste aus DAOs zu einer Liste von DTOs \\
	\hline
	\textbf{Art}    				& Unit-Test \\
	\hline
	\textbf{Vorbedingungen}    		& Das Backend läuft. \\
	\hline
	\textbf{Konfiguration}   	 	& 
	\begin{itemize}
		\item Eine Liste von DAO-Objekten ist erstellt, und die Objekte sind mit Daten gefüllt.
	\end{itemize} \\
	\hline
	\textbf{Ablauf}    				& 
	\begin{enumerate}
		\item Die Funktion listToMqInDto() ausführen, mit der Liste aus DAO-Objekten als Parameter.
		\item Die erhaltenen Werte werden mit der Liste aus DAO-Objekten verglichen.
	\end{enumerate} \\
	\hline
	\textbf{Erwartetes Ergebnis}    & Es gibt keine Unterschiede zwischen den generierten DTO- und den bereits existierenden DAO-Objekten. \\
	\hline
\end{longtable}\label{tab:testfall-M4}


\subsection{Testfälle der Erweiterung Filter}

\begin{longtable}{p{.3\textwidth}|p{.65\textwidth}}
	\hline
	\textbf{Testfall}               & \textbf{M5} \\
	\hline
	\textbf{Beschreibung}   		& Datenbank-Abfrage für das Filtern von einem Filter \\
	\hline
	\textbf{Art}    				& Integration-Test \\
	\hline
	\textbf{Vorbedingungen}    		& Das Backend läuft und eine passende Datenbank existiert. \\
	\hline
	\textbf{Konfiguration}   	 	& 
	\begin{itemize}
		\item Die Daten für den Aufruf werden erstellt.
		\item Der Filter wird gesetzt.
		\item Die erwarteten Daten werden erstellt.
	\end{itemize} \\
	\hline
	\textbf{Ablauf}    				& 
	\begin{enumerate}
		\item Die Datenbank-Abfrage wird durchgeführt.
		\item Die erhaltenen Daten werden mit den bereits erstellten Daten verglichen.
	\end{enumerate} \\
	\hline
	\textbf{Erwartetes Ergebnis}    & Es gibt keine Unterschiede zwischen den gefilterten Daten aus der Datenbank und den erwarteten Daten. \\
	\hline
\end{longtable}\label{tab:testfall-M5}

\begin{longtable}{p{.3\textwidth}|p{.65\textwidth}}
	\hline
	\textbf{Testfall}               & \textbf{M6} \\
	\hline
	\textbf{Beschreibung}   		& Datenbank-Abfrage für das Filtern von mehreren Filtern \\
	\hline
	\textbf{Art}    				& Integration-Test \\
	\hline
	\textbf{Vorbedingungen}    		& Das Backend läuft und eine passende Datenbank existiert. \\
	\hline
	\textbf{Konfiguration}   	 	& 
	\begin{itemize}
		\item Die Daten für den Aufruf werden erstellt.
		\item Die Filter werden gesetzt.
		\item Die erwarteten Daten werden erstellt.
	\end{itemize} \\
	\hline
	\textbf{Ablauf}    				& 
	\begin{enumerate}
		\item Die Datenbank-Abfrage wird durchgeführt.
		\item Die erhaltenen Daten werden mit den bereits erstellten Daten überprüft.
	\end{enumerate} \\
	\hline
	\textbf{Erwartetes Ergebnis}    & Es gibt keine Unterschiede zwischen den gefilterten Daten aus der Datenbank und den erwarteten Daten. \\
	\hline
\end{longtable}\label{tab:testfall-M6}

\begin{longtable}{p{.3\textwidth}|p{.65\textwidth}}
	\hline
	\textbf{Testfall}               & \textbf{M7} \\
	\hline
	\textbf{Beschreibung}   		& Mapping von MqInFilterDto zu MqTableFilterQuery \\
	\hline
	\textbf{Art}    				& Unit-Test \\
	\hline
	\textbf{Vorbedingungen}    		& Das Backend läuft. \\
	\hline
	\textbf{Konfiguration}   	 	& 
	\begin{itemize}
		\item Die Filter werden in einem MqInFilter-Objekt gesetzt.
	\end{itemize} \\
	\hline
	\textbf{Ablauf}    				& 
	\begin{enumerate}
		\item Die Funktion toMqTableFilterQuery() wird ausgeführt mit, den Filtern als Parameter.
		\item Die erhaltenen Daten werden mit den bereits erstellten Filtern überprüft.
	\end{enumerate} \\
	\hline
	\textbf{Erwartetes Ergebnis}    & Es gibt keine Unterschiede zwischen den erstellten und den gemappten Filtern. \\
	\hline
\end{longtable}\label{tab:testfall-M7}

\newpage