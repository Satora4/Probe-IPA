\chapter{Planen}\label{ch:planen}
Dieses Kapitel zeigt die in der IPERKA-Phase «Planen» durchgeführten Arbeiten auf. In dieser Phase werden basierend auf den Anforderungen Arbeitspakete definiert und in einem Zeitplan auf die zehn Probe-IPA Tage aufgeteilt. Zudem werden Lösungskonzepte für die Umsetzung der Seite Business Daten erarbeitet und ein Testkonzept erstellt.

\section{Arbeitspakete}
Die gesamte Arbeit der Probe-IPA wird in Arbeitspakete aufgeteilt. Die Arbeitspakete bestehen aus einer zugehörigen Nummer, einem Namen, dem geschätzten Aufwand und ein erwartetes Ergebnis. In den geschätzten Aufwand sind Zeitreservern mit einberechnet, sodass unvorhergesehenes kompensiert werden kann. Da der Zeitplan in 2-Stunden-Blöcke aufgeteilt ist, wird der geringste Aufwand 2 Stunden sein und im zweier Takt nach oben gehen.

Die Arbeitspakete sind nach der Projektmanagementmethode IPERKA gegliedert. Probe-IPA-spezifische Arbeiten, wie die Expertenbesuche oder das Erstellen des Anhangs, die ausserhalb der eigentlichen Projekts stehen, werden unter «Rahmenaufgaben» aufgeführt.
\include{teil_2/arbeitspakete}

\section{Lösungskonzept für die Struktur vom Backend}
In diesem Abschnitt wir das Lösungskonzept für einen Teil der unter \ref{ch:minimalanforderungen} definierten Anforderungen beschrieben.

\subsection{Erstellung der Endpunkte}
Im Backend existieren noch keine Endpunkte für die Message-Queues, um sie im Frontend darzustellen. Deshalb müssen diese neu implementiert werden. Die Endpoints werden in einer Service-Klasse erstellt, welche von einem Interface implementiert wird. Die Standardfunktionen, wie GET oder UPDATE, sind bereits im Interface definiert und müssen so nur noch in der Service-Klasse implementiert werden. Der Grund für das Interface ist, dass das Interface für die Service-Klasse und die Mock-Service-Klasse genutzte werden kann und beide Klassen die gleichen Funktionen haben aber mit einer unterschiedlichen Implementierung.
Der Zugriff auf die Datenbank erfolgt in der Klasse ...

\newpage