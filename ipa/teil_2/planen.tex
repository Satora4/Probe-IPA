\chapter{Planen}\label{ch:planen}
Dieses Kapitel zeigt die in der IPERKA-Phase «Planen» durchgeführten Arbeiten auf. In dieser Phase werden basierend auf den Anforderungen Arbeitspakete definiert und in einem Zeitplan auf die zehn Probe-IPA Tage aufgeteilt. Zudem werden Lösungskonzepte für die Umsetzung der Seite Business Daten erarbeitet und ein Testkonzept erstellt.

\section{Arbeitspakete}
Die gesamte Arbeit der Probe-IPA wird in Arbeitspakete aufgeteilt. Die Arbeitspakete bestehen aus einer zugehörigen Nummer, einem Namen, dem geschätzten Aufwand und ein erwartetes Ergebnis. In den geschätzten Aufwand sind Zeitreservern mit einberechnet, sodass unvorhergesehenes kompensiert werden kann. Da der Zeitplan in 2-Stunden-Blöcke aufgeteilt ist, wird der geringste Aufwand 2 Stunden sein und im zweier Takt nach oben gehen.

Die Arbeitspakete sind nach der Projektmanagementmethode IPERKA gegliedert. Probe-IPA-spezifische Arbeiten, wie die Expertenbesuche oder das Erstellen des Anhangs, die auserhalb der eigentlichen Projekts stehen, werden unter «Rahmenaufgaben» aufgeführt.

\subsection{Informieren}
Folgende Arbeitspakete gehören zu der IPERKA-Phase «Informieren».

\begin{longtable}{p{.3\textwidth}|p{.65\textwidth}}
	\hline
	\textbf{Nummer}                 & \textbf{1.1}            \\
	\hline
	\textbf{Name}   				& Projektumfeld analysieren und beschreiben                  \\
	\hline
	\textbf{Geschätzter Aufwand}    & 2h                                    \\
	\hline
	\textbf{Erwartetes Ergebnis}    & Die Aufgabestellung ist beschrieben und für den Lernenden der Auftrag klar. Der Lernende kennt das Projektumfeld und hat es dokumentiert.                                    \\
	\hline
\end{longtable}\label{tab:informieren-1.1}

\begin{longtable}{p{.3\textwidth}|p{.65\textwidth}}
	\hline
	\textbf{Nummer}                 & \textbf{1.2}            \\
	\hline
	\textbf{Name}   				& Anforderungen definieren                  \\
	\hline
	\textbf{Geschätzter Aufwand}    & 2h                                    \\
	\hline
	\textbf{Erwartetes Ergebnis}    & Die Anforderungen werden klar definiert und unterteilt in funktionale und nicht funktionale Anforderungen.                                    \\
	\hline
\end{longtable}\label{tab:informieren-1.2}

\subsection{Planen}
Folgende Arbeitspakete gehören zu der IPERKA-Phase «Planen».

\begin{longtable}{p{.3\textwidth}|p{.65\textwidth}}
	\hline
	\textbf{Nummer}                 & \textbf{2.1}            \\
	\hline
	\textbf{Name}   				& Arbeitspakete definieren                  \\
	\hline
	\textbf{Geschätzter Aufwand}    & 4h                                    \\
	\hline
	\textbf{Erwartetes Ergebnis}    & Die gesamten Arbeitspakete sind definiert und nummeriert nach den Phasen von IPERKA.                                    \\
	\hline
\end{longtable}\label{tab:planen-2.1}

\begin{longtable}{p{.3\textwidth}|p{.65\textwidth}}
	\hline
	\textbf{Nummer}                 & \textbf{2.2}            \\
	\hline
	\textbf{Name}   				& Zeitplan erstellen                  \\
	\hline
	\textbf{Geschätzter Aufwand}    & 2h                                    \\
	\hline
	\textbf{Erwartetes Ergebnis}    & Der Zeitplan wird mithilfe der Arbeitspakete erstellt und die bereits erfüllten Aufgaben werden entsprechend markiert.                                    \\
	\hline
\end{longtable}\label{tab:planen-2.2}

\begin{longtable}{p{.3\textwidth}|p{.65\textwidth}}
	\hline
	\textbf{Nummer}                 & \textbf{2.3}            \\
	\hline
	\textbf{Name}   				& Lösungskonzept für die Struktur vom Backend                  \\
	\hline
	\textbf{Geschätzter Aufwand}    & 2h                                    \\
	\hline
	\textbf{Erwartetes Ergebnis}    & Ein Lösungskonzept für die Struktur im Backend wird erarbeitet und dokumentiert.                                    \\
	\hline
\end{longtable}\label{tab:planen-2.3}

\begin{longtable}{p{.3\textwidth}|p{.65\textwidth}}
	\hline
	\textbf{Nummer}                 & \textbf{2.4}            \\
	\hline
	\textbf{Name}   				& Lösungskonzept für die Struktur vom Frontend                  \\
	\hline
	\textbf{Geschätzter Aufwand}    & 2h                                    \\
	\hline
	\textbf{Erwartetes Ergebnis}    & Ein Lösungskonzept für die Struktur im Frontend wird erarbeitet und dokumentiert.                                    \\
	\hline
\end{longtable}\label{tab:planen-2.4}

\begin{longtable}{p{.3\textwidth}|p{.65\textwidth}}
	\hline
	\textbf{Nummer}                 & \textbf{2.5}            \\
	\hline
	\textbf{Name}   				& Lösungskonzept für die Struktur von einer Erweiterung                  \\
	\hline
	\textbf{Geschätzter Aufwand}    & 4h                                    \\
	\hline
	\textbf{Erwartetes Ergebnis}    & Der Lernende entscheidet sich zwischen einer der beiden Erweiterungen und erarbeitet für dieses ein Lösungskonzept und dokumentiert diese anschliessend.                                    \\
	\hline
\end{longtable}\label{tab:planen-2.5}

\begin{longtable}{p{.3\textwidth}|p{.65\textwidth}}
	\hline
	\textbf{Nummer}                 & \textbf{2.6}            \\
	\hline
	\textbf{Name}   				& Testkonzept erstellen                  \\
	\hline
	\textbf{Geschätzter Aufwand}    & 4h                                    \\
	\hline
	\textbf{Erwartetes Ergebnis}    & Ein Testkonzept wird erarbeitet. Die zu schreibenden Tests und Testergebnisse sind definiert.                                    \\
	\hline
\end{longtable}\label{tab:planen-2.6}

\subsection{Entscheiden}
Folgende Arbeitspakete gehören zu der IPERKA-Phase «Entscheiden».

\begin{longtable}{p{.3\textwidth}|p{.65\textwidth}}
	\hline
	\textbf{Nummer}                 & \textbf{3.1}            \\
	\hline
	\textbf{Name}   				& Lösungsvarianten evaluieren                  \\
	\hline
	\textbf{Geschätzter Aufwand}    & 4h                                    \\
	\hline
	\textbf{Erwartetes Ergebnis}    & Mögliche Lösungsvarianten wurden evaluiert und die umzusetzende Lösungsvariante ist definiert.                                    \\
	\hline
\end{longtable}\label{tab:entscheiden-3.1}

\subsection{Realisieren}
Folgende Arbeitspakete gehören zu der IPERKA-Phase «Realisieren».

\begin{longtable}{p{.3\textwidth}|p{.65\textwidth}}
	\hline
	\textbf{Nummer}                 & \textbf{4.1}            \\
	\hline
	\textbf{Name}   				& Endpoints für Mindestanforderungen erstellen                  \\
	\hline
	\textbf{Geschätzter Aufwand}    & 4h                                    \\
	\hline
	\textbf{Erwartetes Ergebnis}    & Die Endpoints für die Mindestanforderungen werden erstellt, sodass das Frontend alle Daten, die es braucht, und nur die, die es braucht, bekommt.                                    \\
	\hline
\end{longtable}\label{tab:realisieren-4.1}

\begin{longtable}{p{.3\textwidth}|p{.65\textwidth}}
	\hline
	\textbf{Nummer}                 & \textbf{4.2}            \\
	\hline
	\textbf{Name}   				& Seite und Tabelle im Frontend erstellen                  \\
	\hline
	\textbf{Geschätzter Aufwand}    & 4h                                    \\
	\hline
	\textbf{Erwartetes Ergebnis}    & Die Seite ist mit der entsprechenden URL und im Menü unter Business Daten → Queue-Messages (eingehend) erreichbar. Die Tabelle ist stimmig ins UI eingebaut und entspricht der tabellarischen Darstellung.                                    \\
	\hline
\end{longtable}\label{tab:realisieren-4.2}

\begin{longtable}{p{.3\textwidth}|p{.65\textwidth}}
	\hline
	\textbf{Nummer}                 & \textbf{4.3}            \\
	\hline
	\textbf{Name}   				& Endpoints für Erweiterung erstellen                  \\
	\hline
	\textbf{Geschätzter Aufwand}    & 4h                                    \\
	\hline
	\textbf{Erwartetes Ergebnis}    & Die Endpoints für die Erweiterung werden erstellt, sodass das Frontend alle Daten, die es braucht, und nur die, die es braucht, bekommt.                                    \\
	\hline
\end{longtable}\label{tab:realisieren-4.3}

\begin{longtable}{p{.3\textwidth}|p{.65\textwidth}}
	\hline
	\textbf{Nummer}                 & \textbf{4.4}            \\
	\hline
	\textbf{Name}   				& Erweiterung in die Tabelle integrieren                  \\
	\hline
	\textbf{Geschätzter Aufwand}    & 4h                                    \\
	\hline
	\textbf{Erwartetes Ergebnis}    & Die Erweiterung wird im Frontend in die Tabelle integriert. Das Design der Erweiterung muss stimmig zu der Tabelle und der Seite eingebaut werden.                                    \\
	\hline
\end{longtable}\label{tab:realisieren-4.4}

\begin{longtable}{p{.3\textwidth}|p{.65\textwidth}}
	\hline
	\textbf{Nummer}                 & \textbf{4.5}            \\
	\hline
	\textbf{Name}   				& ReleaseNotes schreiben                  \\
	\hline
	\textbf{Geschätzter Aufwand}    & 1h                                    \\
	\hline
	\textbf{Erwartetes Ergebnis}    & Für die neue Tabelle werden ReleaseNotes gschrieben, um die Tabelle und die Erweiterung zu beschreiben und erklären.                                    \\
	\hline
\end{longtable}\label{tab:realisieren-4.5}

\subsection{Kontrollieren}
Folgende Arbeitspakete gehören zu der IPERKA-Phase «Kontrollieren».

\begin{longtable}{p{.3\textwidth}|p{.65\textwidth}}
	\hline
	\textbf{Nummer}                 & \textbf{5.1}            \\
	\hline
	\textbf{Name}   				& Tests                  \\
	\hline
	\textbf{Geschätzter Aufwand}    & 6h                                    \\
	\hline
	\textbf{Erwartetes Ergebnis}    & Das geplante Testkonzept wird Umgesetzt und bei Bedarf ergänzt.                                    \\
	\hline
\end{longtable}\label{tab:kontrollieren-5.1}

\begin{longtable}{p{.3\textwidth}|p{.65\textwidth}}
	\hline
	\textbf{Nummer}                 & \textbf{5.2}            \\
	\hline
	\textbf{Name}   				& Codequalität prüfen                  \\
	\hline
	\textbf{Geschätzter Aufwand}    & 2h                                    \\
	\hline
	\textbf{Erwartetes Ergebnis}    & Die Codequalität wird mithilfe von der Jenkins Pipeline überprüft und bei bedarf überarbeitet.                                    \\
	\hline
\end{longtable}\label{tab:kontrollieren-5.2}

\begin{longtable}{p{.3\textwidth}|p{.65\textwidth}}
	\hline
	\textbf{Nummer}                 & \textbf{5.3}            \\
	\hline
	\textbf{Name}   				& Dokumentation finalisieren                  \\
	\hline
	\textbf{Geschätzter Aufwand}    & 8h                                    \\
	\hline
	\textbf{Erwartetes Ergebnis}    & Die Dokumentation ist nachvollziehbar und verständlich. Das Dokument wird nach Schreibfehlern durchsucht und verbessert, sodass zu diesem Zeitpunkt fast bis gar keine mehr übrig sind. Die Struktur ist einheitlich und Unschönheiten wurden behoben.                                    \\
	\hline
\end{longtable}\label{tab:kontrollieren-5.3}

\subsection{Auswerten}
Folgende Arbeitspakete gehören zu der IPERKA-Phase «Auswerten».

\begin{longtable}{p{.3\textwidth}|p{.65\textwidth}}
	\hline
	\textbf{Nummer}                 & \textbf{6.1}            \\
	\hline
	\textbf{Name}   				& Kurzfassung schreiben                  \\
	\hline
	\textbf{Geschätzter Aufwand}    & 2h                                    \\
	\hline
	\textbf{Erwartetes Ergebnis}    & Das Projekt wird in einer Kurzfassung zusammengefasst.                                    \\
	\hline
\end{longtable}\label{tab:auswerten-6.1}

\begin{longtable}{p{.3\textwidth}|p{.65\textwidth}}
	\hline
	\textbf{Nummer}                 & \textbf{6.2}            \\
	\hline
	\textbf{Name}   				& Reflexion schreiben                  \\
	\hline
	\textbf{Geschätzter Aufwand}    & 2h                                    \\
	\hline
	\textbf{Erwartetes Ergebnis}    & Das Projekt wird vom Lernenden reflektiert und dokumentiert.                                    \\
	\hline
\end{longtable}\label{tab:auswerten-6.2}

\subsection{Rahmenaufgaben}
Folgende Arbeitspakete gehören zu den Rahmenaufgaben.

\begin{longtable}{p{.3\textwidth}|p{.65\textwidth}}
	\hline
	\textbf{Nummer}                 & \textbf{7.1}            \\
	\hline
	\textbf{Name}   				& Projektstruktur aufsetzen                  \\
	\hline
	\textbf{Geschätzter Aufwand}    & 2h                                    \\
	\hline
	\textbf{Erwartetes Ergebnis}    & Aufbau der Gerüstes des \LaTeX Berichtes, welches die Titelseite, ein Glossar, Das Quellenverzeichniss und ein Abbildungsverzeichnis beinhaltet. Ein Git Repository wird für die Dokumentation aufgesetzt.                                    \\
	\hline
\end{longtable}\label{tab:auswerten-7.1}

\begin{longtable}{p{.3\textwidth}|p{.65\textwidth}}
	\hline
	\textbf{Nummer}                 & \textbf{7.2}            \\
	\hline
	\textbf{Name}   				& Aufgabenstellung und Rahmenbedingungen beschreiben                  \\
	\hline
	\textbf{Geschätzter Aufwand}    & 2h                                    \\
	\hline
	\textbf{Erwartetes Ergebnis}    & Die Aufgabenstellung und Rahmenbedingungen hat der Lernende verstanden und beschreibt sie nochmals um dies zu bestätigen.                                    \\
	\hline
\end{longtable}\label{tab:rahmenaufgaben-7.2}

\begin{longtable}{p{.3\textwidth}|p{.65\textwidth}}
	\hline
	\textbf{Nummer}                 & \textbf{7.3}            \\
	\hline
	\textbf{Name}   				& Projektmanagementmethode definieren                  \\
	\hline
	\textbf{Geschätzter Aufwand}    & 2h                                    \\
	\hline
	\textbf{Erwartetes Ergebnis}    & Eine Projektmanagementmethode wird definiert und durch einer anderen wird erleutert warum sie gewählt wird.                                    \\
	\hline
\end{longtable}\label{tab:rahmenaufgaben-7.3}

\begin{longtable}{p{.3\textwidth}|p{.65\textwidth}}
	\hline
	\textbf{Nummer}                 & \textbf{7.4}            \\
	\hline
	\textbf{Name}   				& Expertenbesuche                  \\
	\hline
	\textbf{Geschätzter Aufwand}    & 2h                                    \\
	\hline
	\textbf{Erwartetes Ergebnis}    & Die Expertenbesuche werden erfolgreich geplant und durchgeführt. Wichtige Informationen bezüglich der Besuche werden an passenden Plätzen erwähnt und dokumentiert.                                    \\
	\hline
\end{longtable}\label{tab:rahmenaufgaben-7.4}

\begin{longtable}{p{.3\textwidth}|p{.65\textwidth}}
	\hline
	\textbf{Nummer}                 & \textbf{7.5}            \\
	\hline
	\textbf{Name}   				& Anhang erstellen                  \\
	\hline
	\textbf{Geschätzter Aufwand}    & 4h                                    \\
	\hline
	\textbf{Erwartetes Ergebnis}    & Ein Anhang mit allen gemachten Änderungen am Quellcode ist dokumentiert und klar von dem unveränderten Code unterscheidbar.                                    \\
	\hline
\end{longtable}\label{tab:rahmenaufgaben-7.5}

\newpage